
\documentclass{beamer}
\usetheme{metropolis}
\usepackage{graphicx}
\graphicspath{{images/}}

\title{Dynamic Searchable Encryption in Very-Large Databases : Data Structures and Implementation}
\subtitle{David Cash, Joseph Jaeger, Stanislaw Jarecki, Charanjit Jutla, Hugo Krawczyk, Marcel-Catalan Rosu and Michael Steiner}

\author{Sajin Sasy}
\institute{University of Waterloo}


\newcommand{\PRFReal}{\mathsf{PRFReal}}
\newcommand{\PRFRand}{\mathsf{PRFRand}}
\newcommand{\RCPAReal}{\mathsf{RCPAReal}}
\newcommand{\RCPARand}{\mathsf{RCPARand}}
\newcommand{\indrcpa}{\mathsf{ind-rcpa}}

% Basic crypto requirements : 
\newcommand{\ooo}{\mathsf{\{0|1\}}}
\newcommand{\id}{\mathsf{id}}
\newcommand{\A}{\mathsf{A}}
\newcommand{\Adv}{\textbf{Adv}}
\newcommand{\F}{\mathsf{F}}

% Basic crypto primitives and definition stuff :
\newcommand{\ssecor}{\mathsf{sse-cor}}
\newcommand{\SSECor}{\mathsf{SSECor}}
\newcommand{\PRF}{\mathsf{PRF}}
\newcommand{\prf}{\mathsf{prf}}
\newcommand{\Enc}{\mathsf{Enc}}
\newcommand{\Dec}{\mathsf{Dec}}

% SMALL Mathsf letters :
% NOTE: conflict for \i and \d with Latex , hence imbued small s before them !
\newcommand{\si}{\mathsf{i}}
\newcommand{\sd}{\mathsf{d}}
\newcommand{\sw}{\mathsf{w}}
\newcommand{\sm}{\mathsf{m}}

% CAPITAL Mathsf letters :
\newcommand{\K}{\mathsf{K}}
\newcommand{\N}{\mathsf{N}}
\newcommand{\V}{\mathsf{V}}
\newcommand{\W}{\mathsf{W}}

% SSE-specific : 
\newcommand{\SSE}{\mathsf{SSE}}
\newcommand{\DB}{\mathsf{DB}}
\newcommand{\EDB}{\mathsf{EDB}}
\newcommand{\Setup}{\mathsf{Setup}}
\newcommand{\Search}{\mathsf{Search}}
\newcommand{\Update}{\mathsf{Update}}
\newcommand{\op}{\mathsf{op}}
\newcommand{\add}{\mathsf{add}}
\newcommand{\del}{\mathsf{del}}
\newcommand{\editp}{\mathsf{edit^+}}
\newcommand{\editm}{\mathsf{edit^-}}




\begin{document}
\maketitle
\section{Introduction}

\begin{frame}{Introduction}
	TBDL : Introduction slides 
\end{frame}

\section{Definitions and Tools}
\begin{frame}{$\PRF$}
\begin{definition}[1] An algorithm $\F$ is a variable-input-length pseudorandom function if for all efficient $\A$, the function
\begin{center}
$\Adv^\prf_{\F,\A} = \Pr$[$\PRFReal^\A_\F(\lambda)=1$] - $\Pr$[$\PRFRand^\A_\F(\lambda)=1$]
\end{center}
is a negligible function
\end{definition}
\vfill
\end{frame}

\begin{frame}{Encryption Schema $\pi$}
\begin{definition}[2] An encryption schema $\pi$ = ($\Enc,\Dec$) has pseudorandom ciphertexts under chosen-plaintext attack if for all efficient $\A$, the function  
\begin{center}
$\Adv^\indrcpa_{\pi,\A} = \Pr$[$\RCPAReal^\A_\pi(\lambda)=1$] - $\Pr$[$\RCPARand^\A_\pi(\lambda)=1$]
\end{center}
is a negligible function
\end{definition}
\vfill
\end{frame}

\begin{frame}{$\PRFReal,\PRFRand,\RCPAReal,\RCPARand$}
\begin{figure}[h]
\centering
\includegraphics[width=10cm, height=2.5cm]{fig2} \\
\caption{D.Cash's $\PRFReal,\PRFRand,\RCPAReal,\RCPARand$}
%TO-DO : Insert fig2 here and FIND fig for tradition PRF and CPA ?Be careful with this.
\end{figure}
\includegraphics[width=10cm, height=2.5cm]{fig5}
\end{frame}

\begin{frame}{Notations}
\begin{itemize}
\item A database is $\DB$ = $(\id_\si,\W_\si)^\sd_{\si=1}$ 
is a $\sd$-tuple of identifier/keyword-set pairs, where $\id_\si \in \ooo^\lambda$ and $\W_\si \subseteq \ooo^*$
\item $\W = \cup^\sd_{\si=1} \W_\si$
\item For a keyword $\sw$, $\DB(\sw)$ is {$\id_\si: \sw \in \W_si$}
\item $\sm = |\W|$ and $\N = \sigma_{\sw \in \W}|\DB(\sw)|$
\end{itemize}
\end{frame}

\begin{frame}{$\SSE$ Definition}
A dynamic searchable symmetric encryption scheme $\pi$ consists of an algorithm $\Setup$ and protocols $\Search$ and $\Update$ between the client and server\footnote{Static is the same without the $\Update$}.
\begin{itemize}
\item $(\K,\EDB) \leftarrow \Setup(\DB)$
\item $\Search$ : client takes $\K$ and query $\sw \in \ooo^{\lambda}$, server takes  as input $\EDB$. The server outputs a set of identifiers and the client has no output.
$(\V;\tau) \leftarrow \Search(\K,\sw,\EDB)$
\item $\Update$ : client takes as input $\K$, an operation $\op \in \{\add,\del,\editp,\editm\}$, a file identifier $\id$ and a set $\W_\id$ of keywords.
$(\EDB';\tau) \leftarrow \Search(\K,\op,\id,\EDB)$
\end{itemize}
\end{frame}

\begin{frame}{$\SSECor$}
\begin{definition}[3] An SSE schema is correct if for all efficient $\A$ : 
\begin{center}
$\Adv^\ssecor_{\pi,\A} = \Pr$[$\SSECor^\A_\pi(\lambda)=1$]
\end{center}
is a negligible function\footnote{Adversary is restricted to never add a duplicate identifier, add redundant keyword to an existing identifier, delete with currently non-existent identifier or delete a keyword from an identifier that does not match it}.
\end{definition}
\vfill
\end{frame}

\begin{frame}{}

\end{frame}

\section{Static Constructions}

\begin{frame}{}
\end{frame}


\end{document}